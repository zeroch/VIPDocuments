\documentclass[pdf]{beamer}
\usepackage{media9}
\usepackage{graphicx}
\usepackage{multicol}

\mode<presentation>{\usetheme{Warsaw}}
%%preamble
\title{Planning And Controls Team}
\author{Ivan Dario Jimenez \and Ze Che \and Myron Lee \and Abraham Marsen \and Edward Aquilar \and Chenkai Shao}
%%preamble ends

\begin{document}

\newcommand{\myfigure}[3]{
\begin{figure}[h!]
  \centering
  \includegraphics[width=#3\textwidth]{#1}
  \caption{#2}
\end{figure}
}

\begin{frame}
  \titlepage
\end{frame}

 \begin{frame}{Outline}
   \begin{multicols}{2}
    \tableofcontents     
   \end{multicols}
 \end{frame}

\section{Introduction}

\begin{frame}{The General Problem}
  \begin{itemize}
  \item Identify target in environment
  \item Approach Target
  \item Interact in a useful way with it(picking)
  \end{itemize}
\end{frame}

\begin{frame}{Our Role}
  \begin{itemize}
  \item Bridge between environment information and physical robot.
  \item Planning: Generate Output from sensory inputs
  \item Sensor-Robot Interface: Generate useful commands for the robot to follow.
  \end{itemize}
\end{frame}

\begin{frame}{Controls And Planning Work}
  \begin{itemize}
  \item Consolidate Planning Application
  \item Advanced mechanism/pipeline for communication with Planning
    \begin{itemize}
    \item Constant update and r/w
    \end{itemize}
   \item High-Level Algorithm
     \begin{itemize}
     \item Explore efficiency of various algorithm
     \item Collaboration with Mapping through established API
     \end{itemize}
  \end{itemize}
\end{frame}

\section{Background}
\subsection{OpenRave}
\begin{frame}{OpenRAVE}
  \begin{itemize}
  \item Open Source
  \item Allows Sensor Integration
  \item capable of IK, FK, path and grasp planning
  \end{itemize}
  \myfigure{Images/OpenRave.png}{OpenRAVE in action}{0.5}
\end{frame}

\subsection{VREP}
\begin{frame}{VREP}
  \begin{itemize}
  \item Composed of 3 elements:
    \begin{itemize}
    \item Control Mechanisms
    \item Calculation Modules
    \item Scene Objects
    \end{itemize}
  \item Well Documented
  \item Friendly API and interface
  \end{itemize}
\end{frame}

\begin{frame}{VREP features}
  \begin{columns}
    \column{0.5\textwidth}
    \myfigure{Images/VREPf1}{Visual Simulator Features}{1}
    \column{0.5\textwidth}
    \myfigure{Images/VREPf2}{Additional control features}{1}
  \end{columns}
\end{frame}

\section{Component Details}

\subsection{Controls}
\begin{frame}{Youbot Controller}
  \begin{itemize}
  \item Each joint composed of ARM Cortex-M3 controller
  \item PID Control
  \end{itemize}
  \myfigure{Images/controllerDesign.png}{Position Controller in which the$n$ stands for motor positions, $e$ for errors, and other variables are optimized parameters.}{1}
\end{frame}

\begin{frame}{Youbot API}
  \begin{itemize}
  \item Youbot library provides full firmware functionality
   \item Support joint-level control
   \item Encapsulates EtherCAT communication
   \item Manipulator movement within the Cartesian Space is not supported
  \end{itemize}
\end{frame}

\subsection{Software Design}

\subsubsection{Architecture}
\begin{frame}{Goals}
  \begin{itemize}
  \item Test-Driven Application(Hardest Part of Algorithms)
  \item Easily Extendable and Reusable for Future Iterations
  \item Decoupled For independent Library Use
  \end{itemize}
\end{frame}

\begin{frame}{Our Role In The Application}
  \begin{itemize}
  \item Must Be to Able handle planning of several components
  \item Flexibility is a requirement given the variety of inputs
  \end{itemize}
\myfigure{Images/SimpleApplicationDiagram.jpg}{Our Role in the robot}{0.25}
\end{frame}

\begin{frame}{Application Diagram}
\myfigure{Images/SoftwareDesign/OverallApplication}{A UML Diagram Of the Application}{1}
\end{frame}

\subsubsection{Threading}
\begin{frame}{Threading}
  \begin{itemize}
  \item Robotic Arm Planning
  \item 2D Map Planning
  \item Update Sensory Information
  \item Constantly Update
  \end{itemize}
\myfigure{Images/MultithreadedProcess.png}{}{0.25}
\end{frame}

\subsubsection{User Interface}
\begin{frame}{User Interface}
  \begin{itemize}
  \item Algorithm Selection(On Same Data)
  \item Select Various Maps
  \end{itemize}
\myfigure{Images/SoftwareDesign/GUISample}{A Possible GUI Design}{0.5}
\end{frame}
\subsection{Path-Planning}
\subsubsection{Graph Traversal Algorithms}
\begin{frame}{A*}
  \begin{itemize}
  \item Promises:
    \begin{itemize}
    \item Optimality
    \item Completeness
    \end{itemize}
  \item Drawbacks:
    \begin{itemize}
    \item Time complexity grows exponentially with added dimensions.
    \item May be too slow for a constant stream of data
    \end{itemize}
  \end{itemize}
  \begin{columns}
    \column{0.5\textwidth}
    \myfigure{Images/planningProblem}{A simple planning problem}{0.5}
    \column{0.5\textwidth}
    \myfigure{Images/planningSolution.png}{A possible solution}{0.5}
  \end{columns}
\end{frame}

\begin{frame}{D*}
  \begin{itemize}
  \item Promises:
    \begin{itemize}
    \item Optimality
      \item Completeness
        \item better at handling a new information
    \end{itemize}
  \item DrawBacks:
    \begin{itemize}
    \item Based on A* therefore same problem with added dimensions
    \end{itemize}
  \end{itemize}
\myfigure{Images/DStarMap}{A map generated with D*}{.25}
\end{frame}

\subsubsection{Random-Sampling Algorithms}
\begin{frame}{Rapidly-Randomly Trees}
  \begin{itemize}
  \item No Promises
  \item Work at decent speeds at higher dimensions
  \end{itemize}
  \myfigure{Images/SamplePlanning/RRTgraph1.png}{What A Rapidly-Randomly Exploring Tree Looks Like}{0.25}
\end{frame}

\begin{frame}{Bidirectional RRT}
  \begin{itemize}
  \item Much faster than traditional RRT's
  \item Still no Promises of Optimality
  \end{itemize}
  \myfigure{Images/SamplePlanning/BidirectionalRRT.png}{A Bidirectional RRT}{0.25}
\end{frame}

\begin{frame}{RRT*}
  \begin{itemize}
  \item Promises to be asymptotic optimal.
  \item rewires the tree to accept new useful information
  \item Adds all new points to the tree(higher memory usage)
  \end{itemize}
\myfigure{Images/SamplePlanning/rrtvsrrtstar2}{Comparison between RRT* and RRT's}{0.5}
\end{frame}

\begin{frame}{RRT\#}
  \begin{itemize}
  \item Same Promises as RRT*
  \item Does not keep all points in the tree(only relevant ones)
  \item less memory usage more difficult to implement(no library)
  \end{itemize}
\myfigure{Images/SamplePlanning/RRSharpvsRRTStar.png}{Comparison between RRT* and RRT\#}{0.5}
\end{frame}

\begin{frame}{RRT Comparison}
\includemedia[
  width=0.4\linewidth,
  height=0.3\linewidth,
  activate=pageopen,
  addresource=./Videos/d2pt1rrtstar.MP4,
  flashvars={source=./Videos/d2p1rrtstar.MP4}
]{}{VPlayer.swf}
\end{frame}

\section{Conclusion}
\begin{frame}{Conclusion}
  \begin{itemize}
  \item Focus on Reusable application for future iterations or GTRI
  \item Algorithms are important but not our main focus (we have already implemented many of them)
  \item Extensible application to help test other planning and perception systems. 
  \item A lasting code base
  \end{itemize}
\end{frame}
\section*{References}
\begin{frame}[allowframebreaks]{References}
  \begin{thebibliography}{20}
  \bibitem{VisionRobotics} \href{http://visionrobotics.com/vrc/index.php?option=com_zoom&Itemid=26&catid=6}{Vision Robotics}
  \bibitem{wallye} \href{http://wall-ye.com/}{Wall-Ye}
  \bibitem{roboticcharvest} \href{http://www.roboticharvesting.com/products.html}{Robotic Harvesting}
  \bibitem{kuka} \href{http://www.kuka-labs.com/en/service_robotics/research_education/youbot/}{Kuka Labs}
  \bibitem{robocup} \href{http://www.robocupatwork.org/download/RoboCup-At-Work_Camp_2012/RAW_Camp2012-Jan-Paulus_youBotAPI.pdf}{RoboCup}
  \bibitem{metu} \href{http://www.eee.metu.edu.tr/~ee402/2013/EE402RecitationReport_4.pdf}{Metu}
  \bibitem{Astar} \href{http://en.wikipedia.org/wiki/A*_search_algorithm}{A*}
  \bibitem{AstarTutorial} \href{http://www.policyalmanac.org/games/aStarTutorial.htm}{A* Tutorial}
  \bibitem{AstarCompare} \href{http://theory.stanford.edu/~amitp/GameProgramming/AStarComparison.html}{A* Comparison}
  \bibitem{heuristics} \href{http://www.policyalmanac.org/games/heuristics.htm}{Heuristics}
  \bibitem{dynamicp} \href{http://www.frc.ri.cmu.edu/~axs/dynamic_plan.html}{Dynamic Planning}
  \bibitem{dstar} \href{http://en.wikipedia.org/wiki/D*}{D*}
  \bibitem{RRT} \href{http://msl.cs.uiuc.edu/rrt/papers.html}{Rapidly-Exploring Random Trees}
  \bibitem{RRTBidi} \href{http://people.csail.mit.edu/aperez/obirrt/}{Bidirectional RRTs}
  \bibitem{RRTstar} \href{http://sertac.scripts.mit.edu/web/?page_id=15}{RRT*}
  \bibitem{RRTsharp} \href{http://ieeexplore.ieee.org/xpl/login.jsp?tp=&arnumber=6630906&url=http://ieeexplore.ieee.org/iel7/6615630/6630547/06630906.pdf?arnumber=6630906}{RRT\#}
  \bibitem{kdtree} \href{https://www.cise.ufl.edu/class/cot5520fa09/CG_RangeKDtrees.pdf}{K-D Trees}
    \end{thebibliography}
\end{frame}
\end{document}

